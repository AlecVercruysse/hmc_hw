\documentclass[12pt,letterpaper]{hmcpset}
\usepackage[margin=1in]{geometry}
\usepackage{graphicx}
\usepackage{enumitem} % enumerate
\newcommand{\vb}{\mathbf}
% example usage of amssymb: $\mathbb{Z}$
% amsmath is loaded.

% info for header block in upper right hand corner
\name{}
\class{Math 19 - 07}
\assignment{Homework \#9}
\duedate{11/1/2019}

\begin{document}

\problemlist{Colley 2.2.14, 2.3.22, 2.3.33, 2.3.38, 2.3.42, 2.3.44, 2.4.29(a),(c)}

\begin{problem}[Colley 2.2.14]
  Evaluate the limit or explain why the limit fails to exist:
  \[ \lim_{(x,y) \to (0,0)} \frac{xy}{x^2 + y^2} \]
\end{problem}
\clearpage
\begin{problem}[Colley 2.3.22]
  Find the gradient $\nabla  f(\vb a)$:
  \[ f(x,y) =  e^{xy} + \ln(x-y), \vb a=(2, 1) \]
\end{problem}
\clearpage

\begin{problem}[Colley 2.3.33]
  Find the matrix $Df(\vb a)$ of partial derivatives:
  \[\vb f(s, t) = (s^2, st, t^2), \vb a = (-1, 1) \]
\end{problem}
\clearpage

\begin{problem}[Colley 2.3.38]
  Find an equation for the plane tangent to the graph of $z = 4\cos x y$ at the point $(\pi /3, 1, 2)$.
\end{problem}
\clearpage

\begin{problem}[Colley 2.3.42]
  Suppose that you have the following information concerning a differentiable function $f$:
  \[ f(2,3) = 12, f(1.98,3) = 12.1, f(2,3.01) = 12.2\]
  \begin{enumerate}[label=(\alph*)]
  \item Give an approximate equation for the plane tangent to the graph of $f$ at $(2,3,12)$.
  \item Use the result of part (a) to estimate $f(1.98, 2.98)$.
  \end{enumerate}
\end{problem}
\clearpage

\begin{problem}[Colley 2.3.44]
  \[ f(x,y) = 3 + \cos \pi xy, f(0.98, 0.51)\]
  \begin{enumerate}[label=(\alph*)]
  \item Use the linear approximation $h(\vb x) = \vb f(\vb a) + D\vb f(\vb a)(\vb x - \vb a)$ to approximate the indicated value of the given function $f$.
  \item How accurate is the approximation determined in part (a)?
  \end{enumerate}
\end{problem}
\clearpage
\begin{problem}[Colley 2.4.29(a),(c)]
  The three-dimensional \textbf{heat equation} is the partial differential equation
  \[ k\left( \frac{\partial^2T}{\partial x^2} + \frac{\partial^2T}{\partial y^2} + \frac{\partial^2T}{\partial z^2}\right) = \frac{\partial T}{\partial t}, \]
  where $k$ is a positive constant.
  It models the temperature $T(x, y, z, t)$ at the point $(x, y, z)$ and time $t$ of a
  body in space.
  \begin{enumerate}[label=(\alph*)]
  \item \addtocounter{enumi}{1} We examine a simplified version of the heat equation.
    Consider a straight wire ``coordinatized'' by $x$.
    Then the temperature $T(x, t)$ at time $t$ and position $x$ along the wire is modeled by the one-dimensional heat equation
    \[ k \frac{\partial^2T}{\partial x^2} = \frac{\partial T}{\partial t}.\]

    Show that the function $T(x, t) = e^{-kt} \cos x$ satisfies this equation.
    Note that if $t$ is held constant at value $t_0$, then $T(x, t_0)$ shows how the temperature varies along the wire at time $t_0$.
    Graph the curves $z = T(x, t_0)$ for $t_0$ = 0, 1, 10, and use them to understand the graph of the surface $z = T (x, t)$ for
    $t \geq 0$.
    Explain what happens to the temperature of
the wire after a long period of time.
\item Now show that $T(x, y, z, t)=e^{-kt} (\cos x +
\cos y + \cos z)$ satisfies the three-dimensional heat
equation.
  \end{enumerate}
\end{problem}
\end{document}
