\documentclass[12pt,letterpaper]{hmcpset}
\usepackage[margin=1in]{geometry}
\usepackage{graphicx}
\usepackage{enumitem} % enumerate
% example usage of amssymb: $\mathbb{Z}$
% amsmath is loaded.

% info for header block in upper right hand corner
\name{}
\class{Math 21-01}
\assignment{Homework \#3}
\duedate{9/27/19}

\begin{document}

\problemlist{Bollman 5.2, 5.10, 5.14, 6.2, 6.4}
\begin{problem}[1]
  Suppose you play a fair game where you win or lose \$5 on each bet.
  \begin{enumerate}[label=(\alph*)]
  \item Starting with \$15, use the gambler's ruin formula to find the chances of reaching \$60 before going broke.
  \item Suppose you start with \$15, and you boldly try to reach \$60 by betting everything twice in a row. What are your chances of success?
  \item Answer questions (a) and (b) when your probability of winning each bet is 0.6. (Yes, you can use the gambler’s ruin formula here.) 
  \item For the 0.6 game, suppose your goal is \$80 instead of \$60. If you adopt the bold strategy of betting everything (until you are more than halfway to your goal, when you bet just enough), what is your chance of success?
  \item Same as (d), but now your target is \$90.
  \end{enumerate}
\end{problem}
\begin{solution}

\end{solution}
\pagebreak
\begin{problem}[Bollman 5.2]
  Here’s a combination of chuck-a-luck and craps that has been seen in the craps pit at a number of casinos.
  The player may bet on any single number from 1 to 6 and is paid off according to the number of dice that show that number.
  If two dice show your number, the payoff is 4 to 1; if one die shows your number, the bet is paid off at 2 to 1.
  Find the expected value of a \$1 bet on a single number.
  How does this compare with the HA of more traditional craps bets?
  (Note: The game played uses just two dice.)
\end{problem}
\begin{solution}

\end{solution}
\pagebreak
\begin{problem}[Bollman 5.10]
  \textit{Boule} or \textit{La Boule} is another roulette variation that, as the name suggests, is of French origin.
  The wheel is much simpler, using only the numbers 1 to 9.
  A rubber ball is thrown into the spinning wheel and eventually settles into a hole at the center.
  Each number has four holes allocated to it.
  Numbers 1, 3, 6, and 8 are black; 2, 4, 7, and 9 are red.
  The 5 is yellow and functions somewhat like the green zero in standard roulette.
  The following bets are available at La Boule:
  \begin{center}
    \begin{tabular}{ccc}
      \textbf{Bet} & \textbf{Payoff} & \textbf{Description} \\
      Red/Black    & 1 to 1 & bet on red or black \\
      Low (Manqué) & 1 to 1 & Bet on 1, 2, 3, 4 \\
      High (Passé) & 1 to 1 & Bet on 6, 7, 8, 9 \\
      Odd/Even & 1 to 1 & Bet on odd or even \\
      Single number & 7 to 1 & Bet on any one number
    \end{tabular}
  \end{center}
  The number 5 is neither odd nor even, high nor low.
  Notice that the payoff structure of La Boule is considerably simpler than standard roulette or Royal Roulette–only two different payoffs are available.
  Compute the house advantage on each of the two payoffs.
  How do these HAs compare to those of standard roulette and Royal Roulette?
\end{problem}
\begin{solution}

\end{solution}
\pagebreak
\begin{problem}[Bollman 5.14]
  The \textit{Any Triple} sic bo bet pays off at 30 to 1 if all three dice show the same number. What is the house advantage of this bet?
\end{problem}
\begin{solution}

\end{solution}
\pagebreak
\begin{problem}[Bollman 6.2]
  Some casinos have offered an insurance bet when the dealer’s upcard is a ten-count card, which pays off at 10 to 1 odds if the hole card turns out to be an ace.
  \begin{enumerate}[label=(\alph*)]
  \item Assume that the game is being dealt from a fresh double deck and that the only cards you can see are your hand and the upcard. Calculate the expected value if you hold
    \begin{enumerate}[label=\roman*.]
    \item No aces.
    \item One ace.
    \item Two aces.
    \end{enumerate}
  \item If you have a natural, calculate the expectation of your total wager, assuming that you make an insurance bet for half of your original bet and that blackjack pays off at 3 to 2.
    How does it compare to the expectation without an insurance bet?
  \end{enumerate}
\end{problem}
\begin{solution}

\end{solution}
\pagebreak
\begin{problem}[Bollman 6.4]
  The Four Queens Casino in downtown Las Vegas once offered a “Red/Black” blackjack side bet on the color of the dealer’s upcard.
  The bet paid off at even money, with the provision that if the upcard was a deuce of the color bet, the wager pushed.
  Find the expectation of this bet.
\end{problem}
\begin{solution}

\end{solution}
\pagebreak
\end{document}
